\section{Introduction} \label{sec:introduction}
Named Entity Recognition(NER) is a task of automatically finding Named Entities(NE's) such as persons, locations and organizations in text documents, as well as disambiguating between said types. Different methods have been tried over the last two decades, most of them language dependent. Near-human performing NER systems have already been developed for English, but not for Dutch. NER is generally applied to a specific domain for which the task is optimized, such as finding chemical NE's \cite{rocktaschel2012chemspot}. 

In this paper I want to investigate NER in the domain of Dutch parliamentary items. These parlemntiary items are for example resolutions proposed or letters written by De Eerste Kamer (the Senate) and De Tweede Kamer (the House of Representatives) to the government. A proposed goal system that makes use of the NE's that have been found, allows parties of interest to 'subscribe' to particular entities. These parties will be notified of any new parliamentary items containing these entities. 

A relatively new parser, Frog\footnote{\url{https://languagemachines.github.io/frog/}}, can do a multitude of things in analyzing a text. Frog will be viewed more in depth in section \ref{sec:rel}. So far there haven't been any scientific evaluations of the parser. In this paper I'm going to evaluate the NER-task of Frog for the domain of parliamentary items and compare results of the parser on the CoNLL-2002 dataset, which was used in the CoNLL-2002 shared task\footnote{\url{http://www.cnts.ua.ac.be/conll2002/ner/}} about NER.

I will try to answer the main question by answering the corresponding subquestions:
\begin{enumerate}
    \item What is the best way of finding Named Entities in parliamentary items?
    \begin{enumerate}
        \item   How does Frog parser perform in finding NE's in parliamentary items?  On which areas does Frog perform well? On which areas does it not do well?
        \item How can Frog be improved for the given task?
    \end{enumerate}
\end{enumerate}



\paragraph{Overview of thesis}
At first I want to look at research that has been done on this topic, and take a more in-depth look at the Frog parser and its components. Then I will describe my approach of evaluating the Frog parser on parliamentary items and the way of doing the error analysis. This approach is followed by the results, which will form a conclusion about how well Frog is currently suited for the main task as well as how Frog can be improved for this task. A discussion of approach will conclude this paper.

