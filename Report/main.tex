\documentclass{article}
\usepackage{hyperref}
\usepackage{pdfpages} % http://mirror.unl.edu/ctan/macros/latex/contrib/pdfpages/pdfpages.pdf
\usepackage{booktabs} 
\usepackage{apacite}
\usepackage{multicol}
\usepackage{geometry}
\usepackage{amsmath}
\usepackage{array,ragged2e}
\usepackage[linesnumbered,ruled]{algorithm2e}
\usepackage{wasysym}
\usepackage{listings}
%\geometry{a4paper,pdf, total={170mm,257mm},left=25mm, right=25mm, top=20mm}

\begin{document}


%\setlength{\columnsep}{1cm}



\input{titlepage}


\pagebreak

\tableofcontents

\pagebreak

\begin{abstract}
A dutch natural language processing system, Frog, is used to recognize Named Entities in Dutch parliamentary items, as well is assigning a type to the retrieved Named Entities. Its performance is evaluated by comparing results on the CoNLL-2002 test sets to parliamentary items test set, for which a newly annotated corpus has been constructed. The most difficult task seems to be type prediction of a Named Entity, for which, consequently, a type reclassification step is provided, using majority voting and a parliamentary gazetteer that contains types for the most occurring Named Entities in the domain. These methods have shown an improvement in type  precision, which, together with high recall, makes Frog perfectly suitable for a suggested system that requires all occurring Named Entities per item. 
\end{abstract}


\pagebreak


% Here you input all your sections in seperate files
%\begin{multicols}{2}


\section{Introduction} \label{sec:introduction}
Named Entity Recognition(NER) is a technique used in Information Retrieval, defined as automatically finding Named Entities(NE's) such as persons, locations and organizations in text documents, as well as disambiguating between said types. Different methods have been tried over the last two decades, most of them language dependent, due to the fact that language features are used as indicators, yet grammar and vocabulary differ greatly across languages. While near-human performing NER systems have already been developed for English, this does not appear to be the case for Dutch. The reason for this being that most natural language applications are confined to the language in the field of study, for which English is predominant.  

Although progress has been made, especially for English, several encountered problems are not paired with unanimously accepted solutions. 
One might for example reason that using a large gazetteer, which is actively updated, is useful in the retrieval of entities. While good results have been found using this approach, it, solely, cannot be carried out into large scale applications, because the completeness of the gazetteer and the application runtime are inversely proportional to eachother. Comprehensive gazetteers require more time to search through, even with hashing techniques, while small gazetteers will affect recall negatively. 
Secondly, ambiguity introduces itself as the principal difficulty for type assignment: An entity can be of different role, albeit completely identical in appearance. Humans are able to quite easily disambiguate an entity, even though eludication of choice can be difficult. Therefore computer systems are set up with the arduous task of automating this process.

NER is generally applied to a specific domain for which the task is optimized, such as finding chemical NE's \cite{rocktaschel2012chemspot}. 
In this paper NER is applied in the domain of Dutch parliamentary items, which are for example resolutions proposed or letters written by De Eerste Kamer (the Senate) and De Tweede Kamer (the House of Representatives) to the government. A suggested goal system that utilizes the acquired NE's, notifies parties of interest of any new, relevant parliamentary items.
The answer to the question of how to effectively acquire these NE's with corresponding type for the suggested system will be sought for.
This is an improvement to a baseline string search, because unknown strings are not retrieved. Nor do the retrieved strings contain meaning, as they hold no type information.

A relatively new parser, Frog\footnote{\url{https://languagemachines.github.io/frog/}}, that has been maintained, can apply a multitude of natural language processing techniques on a text. Frog will be examined thoroughly in section \ref{sec:rel}. Hitherto no scientific evaluations of the latest iteration of the system have been published. For this reason an evaluation of  the NER-task of Frog for the domain of parliamentary items has been performed and results are compared to those achieved on the CoNLL-2002 dataset, which was used in the CoNLL-2002 shared task\footnote{\url{http://www.cnts.ua.ac.be/conll2002/ner/}}. Subsequently, a method will be shown for the improvement of entity \textit{type} assignment using majority voting and a parliamentary gazetteer.

\paragraph{Overview of thesis}
At first research that has been done on this topic, is looked into, which will provide a more in-depth look at Frog and its NER-module. Secondly the approach of evaluating Frog on parliamentary items is described, as well as the way of doing the error analysis and how improvement of the entity type assignment was attempted. This approach is followed by the results, which will be used to form a conclusion about how well Frog is currently suited for the main task as well as how Frog can be further improved for this task. A discussion of approach will conclude this paper.


%\clearpage
\section{Related work} \label{sec:rel}
Named-entity recognition(NER) comprises two tasks. Retrieval of NE's is regarded as the primary task, and type classification of retrieved NE's as the secondary task \cite{buitinck2012two}. In the paper of Buitinck and Marx an averaged perceptron is used for both stages, each consisting of its own feature parameters. The benefit of two-staged NER, in which a different algorithm per stage is used, lies in the opportunity of optimizing each algorithm individually.  

Recently, the most prevalent implementation of NER applies a supervised machine learning approach. An annotated corpus is required for apprehension of features (for instance syntax and context) that indicate the occurrence of an NE, as well as the type. This is state of the art as opposed to hand-written extraction rules. The downside of learning off an annotated corpus is the domain limitation that the corpus entails. Additionally, a large training corpus is required for supervised learning, along with the fact that the corpus has domain resemblance to the goal domain of the NER-application. Currently, the only Dutch corpus that was made publicly and freely available, is the CoNLL-2002 data set of news items that appeared in the Belgian newspaper \textit{De Morgen}. The data set contains 301,418 tokens with annotated Part-of-Speech(PoS) and IOB-tag\footnote{\url{https://en.wikipedia.org/wiki/Inside_Outside_Beginning}} hyphenated with entity type. IOB, short for inside-outside-beginning, is used to show if a token is the beginning of an entity consisting of multiple tokens; is second or one thereafter in the tag sequence; or outside, meaning not an entity on its own or part of one. When spoken of an \textit{entity} tag in this paper, the complete tag (IOB+type) is referred to, which can be understood from the visualization in figure \ref{fig:iob}.

\begin{figure}
    \centering
    \includegraphics[scale=0.5]{fig/entity_tag}
    \caption{The components of the entity tag}
    \label{fig:iob}
\end{figure}

A variety of classifier types can be chosen from, such as a memory-based learning classifier, a support-vector machine or a conditional random field.
No significant results are achieved by a combination of different classifiers, subsequently classifying based on weighted votes, as is shown by an in-depth evaluation of an ensemble of classifiers for Dutch  \cite{desmet2014fine}. Higher precision and recall are found by the optimization of features in a single classifier.

Likewise an \emph{unsupervised} machine learning approach has been suggested, for which no manual construction of gazetteers or annotated corpora is required \cite{kazama2008inducing}. While gazetteers are considered the most precise for NER, maintaining large dictionaries can be expensive. Clustering of verb-multi-noun dependencies allows for the automatic formation of gazetteers, avoiding said issue.

Frog is an expansion to TADPOLE \cite{bosch2007efficient}. The system is modular, composed of tokenization, lemmatization, chunking and morphological segmentation of word tokens. In addition NE's are detected in Dutch texts using a memory-based tagger \cite{daelemans2010mbt}. Frog is designed for high accuracy, fast processing and low memory usage, qualifying itself to process numerous parliamentary items. Frog is trained on the SoNaR corpus consisting of one million annotated and manually verified NER-labels. The NER-module has not been trained on PoS, but on the contextual relation between words and entity tags directly.  The assignment decision for each token is made with a look-up in memory, in which the instance base is stored. Unknown words are classified by a comparison to this instance base. The class of the instance that matches the contextual and internal features of the unknown instance the closest, is chosen. A processed Dutch sentence example is displayed in figure \ref{fig:frog_out}. Each token is associated with its own features, which are delimited by tabs.

\begin{figure}
    \centering
    \includegraphics[scale=0.7]{fig/frog_output}
    \caption{The raw output of a processed Dutch sentence using Frog}
    \label{fig:frog_out}
\end{figure}






%\clearpage
\section{Methodology}
\label{sec:meth}


\subsection{Description of the data}
The dataset is a set of lobby documents scraped from the web. Lobbying can be defined as the act of attempting to influence decisions made by officials in a government, most often legislators or members of regulatory agencies. Lobby documents are for example: resolutions, chamber inquiries, letters to the government and more. A distribution of different types can be seen in figure \ref{fig:data_dis}. The result of scraping these documents is that certain figures or itemization structures are textualized into words concatenated with itemization numbers, table entry titles or something similar. However, Frog's tokenization module will most of the time find the correct tokens.  These documents are exported as consistent JSON dicts, with for each doc corresponding meta information about the document such as the document ID, source, and type. The format can easily be read with Python. For domain reduction purposes, only parliamentary items are included in the data set. These items form the majority of the items and are in itself a mix of all types, except for the news items, in which the named entities vary greatly in domain compared to the Parliamentary Items.

\begin{tabular}{l*{6}{c}r}\label{fig:data-dis}
Type              & Count \\
\hline
Parliamentary item & 131906  \\
News item & 95261 \\
Chamber inquiry & 38804  \\
Voting & 20930 \\
Chamber letter & 19828
Agenda item & 18950  \\
Resolutions & 4792 \\
\end{tabular}

\subsection{Methods}
The following sections will describe the stages required to examine the suitability of Frog for doing NER on Parliamentary Items. I will first describe how Frog has been used. Then I will go over the preparation needed to evaluate Frog on the CoNLL-2002 data set and the set of Parliamentary Items. Lastly I will explain how the reclassification of entity types was performed. 

\subsubsection{Frog usage}
Frog is open-source software that can be modified or redistributed under the GNU General Public License\footnote{\url{http://www.gnu.org/copyleft/gpl.html}}. The results in this paper are acquired by running Frog through a virtual machine on Windows. All necessary dependencies of Frog, including Frog itself, are provided by the LaMachine software distribution\footnote{\url{https://proycon.github.io/LaMachine/}}. Frog can be run through Python using the python-frog binding which is also inlcuded in the virtual machine.

\subsubsection{Evaluation method}
The CoNLL-2002 data set uses the format of figure \ref{fig:conll}. Every line contains three things, each seperated by space: the token, which is an unprocessed word from a sentence, the token its corresponding part-of-speech, and the token its IOB-tag. Frog its Folia output looks similar with tab-delimited column format , but with additional information per token. To release Frog on CoNLL, the test set first has to be reformatted to its original text. This is done by removing the PoS and IOB-tags, concatenating the remaining words into their original sentences thereafter.  Frog chunks together multi-word entities and other words that are closely related on one line in the output. This will make it differ from the CoNLL output, as such the output is splitted with python using regular expressions. Now the processed text has the same line mapping per token as CoNLL.

It needs to be taken into account that the annotation guidelines for the SoNar corpus differ from CoNLL. SoNar has a wider range of entity types, such as EVE for event and PRO for product. These types are annotated in CoNLL as MISC. Therefore the additional types that Frog outputs are mapped to MISC. There is also a guideline difference regarding locational adjectives, such as 'Dutch' or 'European'. CoNLL guidelines annotate such an adjective as MISC, while Frog is trained to assign LOC as type. These LOC outputs of Frog have to be reassigned to MISC as well.

\subsubsection{Reclassification of entity types}
In the case of unknown words the contextual word relations may indicate a type correctly, however, with multiple occurrences of the same unambiguous entity throughout a text, context differs from time to time. As a result an entity can be tagged correctly as a PER 90% of the time, but have an incorrect type assigned for the remaining 10%. This is expected to be resolved using \textbf{majority voting}. In the case when a certain type is dominant over a minority, this minority is reclassified as the dominant type. To retrieve information regarding the dominant types per entity, a train set of thousand parliamentary items has been processed by Frog. The total occurrences of the entity types have been counted over all documents. For each token in the test set that has been assigned to be an entity, we perform majority voting. Whether an entity type is dominant for that entity is decided based on a threshold value. When the fraction of an entity type over the total amount of entity type counts is larger than the threshold value, the type is considered dominant.

Secondly, Frog's entity type knowledge is enriched by feeding it a gazetteer in the domain of parliamentary items. This gazetteer is constructed with a combination of automatic extraction of entities with Frog and manual annotation of the correct type. The automatic extraction is performed on the training set of thousand parliamentary items. The extracted entities are sorted on popularity, giving annotation  priority to the most common entities.



%\clearpage
\section{Evaluation results}
Subsection \ref{subsec:per_mes} contains the regular performance evaluation measures, such as recall, precision and F-measure. The example in subsection \ref{subsec:eval_example} will clarify how the performance measures have been obtained. The type prediction distribution in subsection \ref{subsec:type_pred} shows the varying difficulty in predicting specific types.

\label{sec:eva}
\subsection{Performance measures}\label{subsec:per_mes}
The performance evaluation for the different test sets can be seen in figure \ref{fig:performance}. The CoNLL performance is an average of both the ned.testa -and ned.testb set. The parliamentary item set is of the same size as ned.testa. The last column in the figure shows performance after reclassification(\ref{subsec:reclas}) of entity types in the parliamentary items set. For all measures evaluation is done \textbf{per token}. Accuracy is obtained by binarily evaluating all tokens: if a token is predicted as an NE (regardless of type), but it turns out not to be, this token prediction is marked as incorrect. Tokens not seen as an entity while they are, idem. Low accuracy would be the result of Frog incorrectly identifying non-entities as entities, since the other way around would not hit accuracy hard due to the sparcity of entities in both test sets. Recall is measured as the proportion of retrieved tokens tagged as an entity that were annotated in the test set, again not looking at entity type. IOB-precision is the precision of the first part of the entity tag(figure \ref{fig:entity_tag}). This will measure how precisely the beginning of an entity consisting of multiple tokens has been found, or how succesfully different entities following each other have been seperated. Type precision is decided on the second part of the entity tag.  Regular precision is the combination of type-precision en IOB-precision. F-measure is the harmonic mean of the regular precision and the recall.

\subsection{Evaluation example}\label{subsec:eval_example}
To examplify the performance measures, a single sentence has been evaluated in figure \ref{fig:eval_example}. The accuracy is $\frac{9}{10}$, because \textit{S.A.M.} was not predicted as an entity. This miss is also noticeable in recall, to which a score of $\frac{4}{5}$ is assigned. Type- and IOB-precision are only calculated for the recalled entities. Assigning the wrong type to \textit{Kabeljauwherstelplan} makes for a type-precision of $\frac{3}{4}$. Recognizing \textit{Dijksma} as the first instead of the second token of the person entity determines the IOB-precision at $\frac{3}{4}$. The last column shows the evaluation for the regular precision, which is $\frac{2}{4}$, as both type and IOB need to be correct.

\begin{figure}
\begin{tabular}{l||r|r|r|}
     & \multicolumn{1}{l|}{\textbf{CoNLL}} & \multicolumn{1}{l|}{\textbf{Parliamentary items}} & \multicolumn{1}{m{4cm}|}{\textbf{Parliamentary items + reclassification}} \\\hline \hline
Accuracy & 0.950 & 0.973 & 0.974 \\\hline
Recall & 0.845 & 0.906 & 0.908\\\hline 
IOB-precision & 0.915 & 0.890 & 0.890 \\\hline 
Type-precision & 0.672 & 0.616 & 0.662\\\hline 
Precision & 0.629 & 0.563 & 0.603\\\hline
F-measure & 0.720 & 0.694 & 0.725\\\hline
 \end{tabular}
\caption{Performance overview of both sets}
\label{fig:performance}
\end{figure}

\begin{figure}
\begin{tabular}{|l||l|l|l|l|l|l|l|}
\multicolumn{6}{l}{\large{S.A.M. Dijksma vergaderde over het Kabeljauwherstelplan in Ter Heijde.}} \\\hline
\multicolumn{1}{|l|}{\textbf{Token}} & \multicolumn{1}{l|}{\textbf{Predicted}} & \multicolumn{1}{l|}{\textbf{Correct}} & \textbf{Recalled} & \textbf{Type} & \textbf{IOB} & \textbf{Precise} \\\hline \hline
S.A.M. & O & B-PER & $\times$ & N/A & N/A & N/A \\\hline 
Dijksma & B-PER & I-PER & $\checked$ & $\checked$ & $\times$ & $\times$\\\hline 
vergaderde & O & O & N/A & N/A & N/A & N/A\\\hline 
over & O & O & N/A & N/A & N/A & N/A\\\hline  
het & O & O & N/A & N/A & N/A & N/A\\\hline 
Kabeljauwherstelplan & B-ORG & B-MISC & $\checked$ & $\times$ & $\checked$ & $\times$\\\hline 
in & O & O & N/A & N/A & N/A & N/A\\\hline 
Ter & B-LOC & B-LOC & $\checked$ & $\checked$ & $\checked$ & $\checked$\\\hline 
Heijde & I-LOC & I-LOC & $\checked$ & $\checked$ & $\checked$ & $\checked$\\\hline 
. & O & O & N/A & N/A & N/A & N/A\\\hline
\end{tabular}
\caption{Evaluation example}
\label{fig:eval_example}
\end{figure}

\subsection{Type prediction}\label{subsec:type_pred}
Since type-precision has appeared to be relatively low, it would be interesting to analyze errors made per type. Figure \ref{fig:confusion} shows a prediction distribution for each type. 
It can be seen that location as a type is predicted very accurately, while prediction of the type person seems to be significantly more difficult. A plausible explanation for this could be the effectiveness of gazetteers per type. Location names such as \textit{Den Haag}, \textit{Amsterdam} and \textit{Nederland} appear very frequent in parliamentary items and are more easily covered by a plain gazetteer, while person names are incredibly hard to cover in the same way. Ambiguity is, inter alia, caused by abbreviations. Aditionally, locations are almost never abbreviated. Organizations, however, especially political organs often appear in shortened form. 

\begin{figure}
    \centering
    \includegraphics[scale=0.4]{fig/confusion_matrix_reclassified}
    \caption{Prediction distribution per type}
    \label{fig:confusion}
\end{figure}

%\clearpage
\section{Conclusions}
\label{sec:conc}
Frog has turned out to be best available option in performing NER on Dutch texts.
The high recall of 91\% on the parliamentary items test set indicates that Frog is very capable in retrieving all relevant NE's in this domain. Type assignment, however, has appeared to be substantually more difficult, yielding an overall 62\% precision. As a solution, reclassification of entity types has been proposed, and has proven to  increase overall type precision up to 66\%. This boost in type precision, using the parliamentary gazetteer as auxilliary, makes Frog even more suitable for the NER task of the suggested system, even though high recall is most benificient, considering incorrect type assignment can be circumvented by alternating the search query to include additional types (\ref{app:alt_query}). 

Generally high recall of entities is the most important aspect of an any NER system, because the type assignment can be done in a seperate stage and individually optimized, of which the reclassification step in section \ref{subsec:reclas} is an example.  
A semi-automatic gazetteer can not only be constructed for parliamentary items, but for any given domain. Additionally majority voting can be carried out, which is also domain-independent. Comparable techniques for type disambiguation, such as clustering based on wikipedia pages \cite{cucerzan2007large},  allow Frog to be exported to other Dutch NER tasks.

\section{Discussion of approach}
\label{sec:disc}
Unfamiliarity with installing different external libraries and dependencies in order to employ existing NER systems has demanded sizeable time investment. Frog, however, has been an exception due to the possibility of using a virtual machine that is furnished with all necessary dependencies and libraries. \\
Using Python as programming language certainly has favored specific tasks such as reformatting output, reading and writing from a file, as well as string manipulation using regular expressions. Additionally, the use of JSON dictionaries to save counts and extracted entities per document -together with the provided JSON data- has facilitated data processing.

While all of the tokens in the test set can be used in the determination of precision as performance measure in the PoS-task, only a fraction of the tokens (the NE's) can be used in the NER task. Therefore it seems reasonable that a larger test set is required to achieve higher performance measure assurance. Low density of NE's, of which many also occur multiple times, might not have challenged Frog to an ultimate extent. 

\section{Acknowledgements}
\textbf{Maarten Marx} has supervised me over the course of this bachelor thesis and given intermediate feedback\\
\textbf{Justin van Wees} has supervised me and provided the data\\
\textbf{Maarten van Gompel} has helped me with the set-up of python-frog\\
\textbf{Ko van der Sloot} has provided me with examples and tips of how to use Frog\\
\textbf{Antal van den Bosch} has answered some questions about the memory-based tagger of the NER-module of Frog\\



% your refs

\clearpage
\nocite{daelemans2004timbl}
\nocite{bosch1996morphological}
\nocite{daelemans1999forgetting}
\nocite{nadeau2006unsupervised}
\nocite{zhou2002named}
\nocite{graus2014semanticizing}
\nocite{ratinov2009design}

\bibliographystyle{apacite}
\bibliography{references}

\clearpage
\appendix
\section{Suggested system} \label{app:sug_sys}
\subsection{System overview}
The system in mind requires three variable inputs:
\begin{enumerate}
\item The dump file obtained after running Frog with all NE's per item
\item The original data file with the documents (because the source isn't included in the dump file)
\item The search query
\end{enumerate}
A user of the system has two options: either acquire the NE's in a selected document (without necessary prior knowledge of its contents), or find a document that is most relevant to the search query of the user. The query has to be of the following format: $$TYPE{\text{:}}NE(+TYPE{\text{:}}NE)^n$$ An NE can be negated by putting a '!' right before the term. The result of running the code below using the default parameters can be seen in figure \ref{fig:sys_out}.

\subsubsection{Alternating the search query}\label{app:alt_query}
If the company \textit{Coca-Cola} is not classified as ORG, but as a MISC instead, the following search query will not retrieve results: $$ORG\text{:}Coca\text{-}Cola$$
However, adding another possible type to the query fixes the issue:
$$ORG\text{:}Coca\text{-}Cola\text{+}MISC\text{:}Coca\text{-}Cola$$
In contrast, when \textit{Coca-Cola} is not \emph{recalled}, there is no alternate query that will retrieve the NE. 

\begin{figure}
    \centering
    \includegraphics[scale=0.8]{fig/sys_out}
    \caption{The command line output running the code of the suggested system}
    \label{fig:sys_out}
\end{figure}

\subsection{Raw code}
\begin{lstlisting}
import json, re, webbrowser, sys, argparse
from operator import itemgetter

def main(dump_path, docs_path, query):
    with open(dump_path) as f:
        dump = json.load(f)

    winning_id = get_document(query, dump)

    with open(docs_path) as g:
        for doc in g:
            doc = json.loads(doc)
            if winning_id == doc['_id']:
                print 'Opening document in webbrowser...'
                webbrowser.open(doc['_source']['url'])    

def get_document(query, dump):
    """
    Opens the document most relevant
    to the query in your webbrowser from its 
    original source.
    """
    relevant_docs = []
    positive = []
    negative = []
    elements = query.split('+')

    print 'Entered query:', query

    ## Seperate the query into a list of wanted and unwanted entities in a doc
    for e in elements:
        entity_type, entity = e.split(':')
        if entity_type[0] == '!':
            negative.append((re.sub('!', '', entity_type), entity))
        else:
            positive.append((entity_type, entity))

    print '\n', 'Looking for documents containing:'
    for entry in positive:
        print entry[1], 'with type', entry[0]

    print '\n', 'Avoiding documents containing:'
    for entry in negative:
        print entry[1], 'with type', entry[0]

    for doc_id in dump:
        if matches_query(dump[doc_id], positive, negative):
            relevant_docs.append(doc_id)

    return most_relevant_doc_id(relevant_docs, query, dump)

def matches_query(doc, positive, negative):
    """
    Returns if the document matches the query
    by checking wanted and unwanted entities
    """
    match = False

    ## Look for a positive match
    for entity_type, entity in positive:
        if entity in doc['entities']:
            for tag in doc['entities'][entity]:
                t = tag.split('-')[1]
                if t == entity_type:
                    match = True
    
    ## If positive, look for a negative match
    if match:
        for entity_type, entity in negative:
            if entity in doc['entities']:
                for tag in doc['entities'][entity]:
                    t = tag.split('-')[1]
                    if t == entity_type:
                        match = False

    return match


def most_relevant_doc_id(relevant_docs, query, dump):
    """
    Returns the document ID with the highest
    matching score.
    """
    matching_scores = []
    for doc in relevant_docs:
        score = calculate_matching_score(dump[doc], query)
        matching_scores.append((doc, score))

    print ''
    for pair in matching_scores:
        print 'Document with ID %d has received a matching score of %d'%(int(pair[0]), pair[1])


    ID, best_score = max(matching_scores, key=itemgetter(1))
    print '\n', 'Document with ID %d is the most relevant according to a matching score of %d'%(int(ID), int(best_score)) 

    return ID
        

def calculate_matching_score(doc, query):
    """
    Returns the document's matching score
    to the query
    """ 
    elements = query.split('+')
    positive = []
    score = 0

    ## Get all wanted entities and their type out of the query
    for e in elements:
        entity_type, entity = e.split(':')
        if not entity_type[0] == '!':
            positive.append((entity_type, entity))

    ## Look for a positive match, and increase matching score accordingly
    for entity_type, entity in positive:
        if entity in doc['entities']:
            for tag in doc['entities'][entity]:
                t = tag.split('-')[1]
                if t == entity_type:
                    score += doc['entities'][entity][tag]

    return score

if __name__ == '__main__':

    p = argparse.ArgumentParser()
    p.add_argument('-dump', type=str, help='path to the dump with extracted entities per doc', default='data/lobby/dump')
    p.add_argument('-docs', type=str, help='path to original unprocessed documents', default='data/lobby/train_documents')
    p.add_argument('-query', type=str, help='Query used to find the desired document', default='ORG:Belastingdienst+!LOC:Europa')
    args = p.parse_args(sys.argv[1:])

    main(args.dump, args.docs, args.query)



 
\end{lstlisting}

% Example
% \includepdf[nup=2x3 , pages=-]{sargent-lecture_slides.pdf}
 
%\end{multicols}
\end{document}
